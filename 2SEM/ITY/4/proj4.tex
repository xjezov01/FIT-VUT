\documentclass[a4paper,11pt]{article}

\usepackage[left=2cm,text={17cm, 24cm},top=3cm]{geometry}
\usepackage[utf8]{inputenc}
\usepackage[czech]{babel}
\usepackage{times}

\bibliographystyle{czplain}

\newcommand{\myuv}[1]{\quotedblbase #1\textquotedblleft}
%\hbadness 10000

\begin{document}
\begin{titlepage}
\begin{center}
{\Huge\textsc{Vysoké učení technické v~Brně}}\\
\bigskip
{\huge\textsc{Fakulta informačních technologií}}\\
\vspace{\stretch{0.382}}
{\LARGE Typografie a~publikování\,--\,4.\,projekt}\\
\medskip
{\Huge Bibiografické citácie}\\
\vspace{\stretch{0.618}}
\end{center}
{\Large \today \hfill Filip Ježovica}
\end{titlepage}

\section{Úvod}
Každý potrebuje čas od času vytvoriť nejaký dokument.
Existuje mnoho počítačových nástrojov pre spracovanie textu. Väčšinou sú to distribúcie s~grafickým rozhraním, kde si môžeme \myuv{všetko naklikať}. Naozaj všetko?

\section{Začiatky s~\LaTeX om}
I~ja som používal rôzne distribúcie. Nie vždy som nadstavil všetko čo som potreboval. Pri hľadaní alternatívy sa mi ponúkol systém \LaTeX.

Prvý pohľad na tento systém môže bežného používateľa vystrašiť, pretože sa nejedná o~zaužívané grafické rozhranie s~množstvom tlačítok. \TeX \ pripomína programovací jazyk, pretože dokument formátujeme pomocou príkazov vkladaných priamo do textu.\cite{CSTUG}
Dokumenty vytvorené systémom \LaTeX \ sú na prvý pohľad úhľadnejšie a~všetko je tam kde má byť.

Dobrým pomocníkom pri začínaní so systémom \LaTeX \ môže byť kniha \myuv{Typografický systém Tex} \cite{Olsak:SystemTex}. Pre nadšencov zahraničnej literatúry odporúčam  svetovo preslávený titul \myuv{A~guide to \LaTeX} \cite{Helmut:LatexGuide}.

\section{\LaTeX \ a~český jazyk}
Pravidlá pre vsádzanie českého a~slovenského textu sú rozdielne ako tie anglické.
Tomu je potrebné prispôsobiť i~\TeX.
Dávnejšie sa používal balík \verb|\usepackage{czech}| spolu s~počešteným prekladačom.
Teraz sa používa balík \verb|\usepackage[czech]{babel}| , pri ktorom netreba používať prekladač cslatex a~je možné prekladať priamo prekladačom latex.\cite{CZLATEX}

Niekedy potrebujeme používať nezlomiteľné medzery namiesto obyčajných práve pre správne vsádzanie českého a~slovenského textu.
Slúži na to príkaz vlna \verb|~|. Pre uľahčenie vznikol program \texttt{VLNA}\,--\,autor Peter Olšák. Tento program automaticky doplní všetky potrebné vlnky.\cite{VLNA}

\section{Žijeme \LaTeX om}
\LaTeX \ sa stal obľúbeným a~je okolo neho početná komunita ľudí. Usporiadajú sa rôzne stretnutia, konferencie. Zhrnutie týchto konferencií môžeme nájsť v~zborníkoch. \cite{ZBORNIK}

Všetky nadobudnuté znalosti tohto systému využijeme i~v~budúcnosti. Veľa študentov používa tento systém práve pri tvorbe svojej bakalárskej práci \cite{Hamrsky:bak:RozoznavaniePisma} alebo diplomovej práci \cite{Cerny:diplom:Latex-MSPP}.

\section{Písmo budúcnosti?}
Na desiatkach škôl sa deti prvých a~druhých ročníkov zúčastňujú skúšobného overovania písma \texttt{Comenia Script}. Toto písmo sa podobá tiskovému písmu. Po čase sa zistilo, že žiaci sa toto písmo učia bez problémov, píšu plynulo a~čitateľne, dokonca píšu s~radosťou.
Uvidíme ako sa nasadenie písma bude vyvíjať ďalej.\cite{PISMO}

\section{Nie je farba ako farba}
Vyladenie farieb na monitoroch a~ďalších perifériách je veľmi dôležité. Dnes sú dostupné sondy na kalibráciu periférii i~s~pokročilým software. Celou kalibráciou nás prevedie sprievodca a~nasleduje uloženie ICC profilu. Takáto kalibrácia patrí k~základom úspešnej práce grafika, či fotoateliéru.\cite{FONT}


\newpage
\bibliography{lit}
\end{document}